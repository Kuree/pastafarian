\documentclass{article}
\usepackage[utf8]{inputenc}
\usepackage{tikz}
\usepackage{amsmath, amssymb}
\usepackage{xcolor}
\usepackage{listings}
\usepackage{url}
\usepackage[autosize]{dot2texi}
\usepackage[margin=1in]{geometry}
\usepackage[ruled, linesnumbered]{algorithm2e}

\definecolor{vgreen}{RGB}{104,180,104}
\definecolor{vblue}{RGB}{49,49,255}
\definecolor{vorange}{RGB}{255,143,102}
\SetKw{Break}{break}
\SetKw{Continue}{continue}


\lstdefinestyle{verilog-style}
{
    language=Verilog,
    basicstyle=\small\ttfamily,
    keywordstyle=\color{vblue},
    identifierstyle=\color{black},
    commentstyle=\color{vgreen},
    numbers=left,
    numberstyle=\tiny\color{black},
    numbersep=10pt,
    tabsize=8,
    morekeywords={unique, logic, always_ff, typedef, enum, property, endproperty, cover}
    literate=*{:}{:}1
}

\title{FSM Extraction Algorithm}
\author{Keyi Zhang}
\begin{document}

\maketitle
\section{Graph Definition for the Circuit}\label{sec:1}
Let the entire circuit design to be represented as a graph, $G(V, E)$. We define $V$
as terminals of a net, and $E$ as relations between two net terminals.
For instance, in Listing~\ref{lst:1}, we have three variables, \texttt{a}, \texttt{b},
and \texttt{c}, hence we have three \textit{named} vertices in the graph: they are named
because they have direct correspondence to Verilog variables. Similar to synthesis,
we create a new vertex for every u-/bi-nary operator. These new vertices are not named
since they are created through expressions. For instance, \texttt{b + c} is represented
as a new vertex connected to both \texttt{b} and \texttt{c}.
For assignment, we introduce additional vertices as
``pass-through'' vertex, i.e. additional vertex that connects two vertices. For instance,
in Listing~\ref{lst:1}, the assignment vertex connects \texttt{a} and the new vertex
created by \texttt{b + c}, as showing in Figure~\ref{fig:1}. This new vertex will have
\texttt{assignment} type.
\begin{lstlisting}[style={verilog-style}, caption={Graph example from Verilog},
    label={lst:1}]
logic [15:0] a;
logic [15:0] b;
logic [15:0] c;

assign a = b + c;
\end{lstlisting}

\begin{figure}
    \centering
    \begin{tikzpicture}
    \input{figure1.tex}
    \end{tikzpicture}
    \caption{Graph representation for Listing~\ref{lst:1}.}
    \label{fig:1}
\end{figure}

To describe the behavior of conditional logic, e.g. \texttt{case} and \texttt{if}, and
clocking block, such as \texttt{always\_ff}. We introduce a new type of edge \texttt{control}
for conditional logic and \texttt{NonBlocking} for non-blocking assignment, as
shown in Listing~\ref{lst:2} and Figure~\ref{fig:2}. Notice that by introducing
control edge, we allow ``influence'' from one vertex to another vertex
even though there is no direct assignment. Mathematically speaking, that means
there is a path from the vertex used for conditional logic to the assigned vertex; in
this case, a path from \texttt{c} to \texttt{a}. We claim without proof that the graph
constructed by this method is a directed simple graph, which implies that any loop
in the graph has to go through at least two vertices.

We define the \texttt{source} of vertex is the function $source(v)$:
\[
    source(v) = \{x \mid x \in \mathcal{V} \cup {v}\quad\text{for all}\quad x \in \mathcal{E}(v) \} \cup \bigcup_{x \not \in \mathcal{V} \cup {v}}
                source(x)
\]
where $\mathcal{V}$ denotes the set of vertices that does not have any incoming edges,
i.e., constants and top-level IO ports, and $\mathcal{E}(v)$ is defined as the complete
set of $(x, v)$ such that $(x, v) \in E$ and $(x, v)$ is not a control edge.
Notice that this is a recursive definition to search the ``ultimate'' source of the
vertex, including itself.

\begin{lstlisting}[style={verilog-style}, caption={Control vertex and non-blocking
    assignment Verilog code},
    label={lst:2}]
input logic clk;
logic [15:0] a;
logic [15:0] b;
logic [15:0] c;

always_ff @(posedge clk) begin
    if (b == c) begin
      a <= c;
    end
end
\end{lstlisting}

\begin{figure}
    \centering
    \begin{tikzpicture}
    \input{figure2.tex}
    \end{tikzpicture}
    \caption{Graph representation for Listing~\ref{lst:2}. Dotted arrow represents edge with control
    type, and dashed arrow represents non-blocking assignment.}
    \label{fig:2}
\end{figure}

To avoid unnecessary edges and reduce graph complexity, the control edge only connects to ``one hierarchy''
inside the code block. Nested \texttt{if} statement traversal requires extra ``hops'' at the control nodes.
Notice that this is consistent with the structural RTL implementation, where \texttt{if} statement is
implemented as multiplexers, since the nested \texttt{if} is expressed as chained multiplexers. 

\section{Finite State Machine (FSM)}\label{sec:fsm}
For the purpose of HDL-based design, we categorize FSM into two catagories:
\begin{itemize}
    \item \textbf{Explicit} FSM. This includes finite state machines where each state
    is assigned explicitly to an integer, such as local parameters or enums.
    \item \textbf{Implicit} FSM. This includes counter-based state machines where
    states are defined in a finite range of values and the state value can decrement
    or increment at any cycle.
\end{itemize}
Although mathematically these two types of FSMs are equivalent: one can always be
converted to another, logic designers may prefer one over another depends on the
problem and coding styles. As a result, when we extract FSM we should consider both
types of FSMs.
\subsection{FSM properties and Its Graph Equivalence}
We claim that there are three properties that holds for any FSM, either implicit
or explicit:
\begin{enumerate}
    \item The FSM state variable has to be a register.
    \item\label{item:1} All the sources to that FSM variable have to be either constant, ot itself,
    or combination of both with binary operators such as addition.
    \item The current value of the FSM state should be used to determined the next
    state.
\end{enumerate}

One caveat of last property is that it excludes the cases where a state machine
advances its state at every cycle regardless of the current state. This only happens
when the FSM is used as a counter and the maximum value of the counter is the maximum
value specified by its bit width. This, however, is trivial to detect, as shown
later.

Given the graph construction described in Section~\ref{sec:1}, we can show the
equivalent following properties in our graph:
\begin{enumerate}
    \item Given vertex $v$, for all its connected vertices $w \in V$ such that
    $(w, v) \in E$, Edge $(w, v)$ is non-blocking and $w$ is assignment.
    \item Given vertex $v$, each element in $source(v)$ has be to either a constant,
    or itself ($v$).
    \item Given vertex $v$, for all simple path $P$ from $v$ to $v$,
    $P = (v_0, v_1, \dots, v_i)$ such that $v_0 = v_i = v$ and $(v_{j-1}, v_{j}) \in E$,
    there exist an edge $e = (v_{j-1}, v_j)$ such that $e$ is a control edge.
\end{enumerate}

It is trivial to show the first properties are equivalent since we create an assignment
vertex for each blocking and non-blocking assignment, then connect that assignment vertex
to the sink vertex, which is the only source of incoming edges to a vertex.

To show that the second properties are equivalent, notice that the $source$ function
is recursive and the base case is the set $\mathcal{V} \cup v$ for a given vertex
$v$. Furthermore, Since $V$ is the set of all constants and IO ports, if $sources(v)$ are itself
or constant, it means that $v$ is not a data-path register.

To show the equivalence of the last properties, notice the introduction of \textit{control}
edge, which indicates the presence of a conditional logic. If a variable's current value
is used to determined the next value, there must be some conditional logic to use the
current value to mux out the next value, hence creating a control edge from that vertex
and form a loop.

For the corner case where the counter advances without comparing with the current
value, all its loop path will not have any control edge. However, the path length
must be 4: self $\rightarrow$ increment/decrement vertex $\rightarrow$ assignment vertex
$\rightarrow$ self.

\subsection{Relaxed FSM Properties}
In some cases where data path are mixed with control path, property~\ref{item:1} may not
hold anymore based on the coding style, as shown in Listing~\ref{lst:3} and Listing~\ref{lst:4}. Although these
two listings are logically equivalent, property~\ref{item:1} will fail on Listing~\ref{lst:4}.

\begin{lstlisting}[style={verilog-style}, caption={Conditional increment counter
    with constant source},
    label={lst:3}]
input logic clk;
input logic [15:0] a;
input logic [15:0] b;
logic [15:0] c;

always_ff @(posedge clk) begin
    if (a == b) begin
      c <= c + 1;
    end
end
\end{lstlisting}

\begin{lstlisting}[style={verilog-style}, caption={Conditional increment counter
    with constant source},
    label={lst:4}]
input logic clk;
input logic [15:0] a;
input logic [15:0] b;
logic [15:0] c;

always_ff @(posedge clk) begin
    c <= c + 16'(a == b);
end
\end{lstlisting}

To accommodate this use case, we can relax property~\ref{item:1} with some heuristics:
when the binary operator is addition or subtraction, one of the operand is itself and the
other operand is a single-bit value (before the bit-width conversion), we ican treat that
single-bit value as constant source.

\subsection{Implicit FSM Detection}
Based on the definition in Section~\ref{sec:fsm}, the only difference between implicit and
explicit FSM is that implicit has a source that combines itself with another constant with
a binary operation. We can use that property to distinguish these two FSMs. To put that
in the graphical terms, given a FSM vertex $v$, there exists a path,
$(v_0, v_1, \dots, v_i, v)$, where $v \in \{v_0, v_1, \dots, v_i\}$ and the path from $v$
to itself does not contain control edge.

\subsection{Coupled FSM Interaction}
When the value of one FSM machine affects another FSM, we call these two FSMs \textit{coupled}
and there are some logic that interacts with these two. For instance, in a controller,
the valid output FSM can only be in valid state if the main FSM is in transaction state.
Finding out these interactions between coupled FSM is very straight-forward given our graph setup.
Given two FSM vertices, $v$ and $w$, if there is a simple path from $v$ to $w$ that contains
a control edge, then $v$ and $w$ are coupled.


\section{Algorithms and Implementation Details}
This section describes algorithms used to extract FSM, the type of each FSM, and their interaction.
It also highlights some implementation details to speed up the computation.

\subsection{FSM Extraction}

The overall algorithm to extract FSMs from a given Verilog file is shown in Algorithm~\ref{alg:1}.

\begin{algorithm}[!tbh]\label{alg:1}
    \KwIn{Verilog file $V$}
    \KwResult{List of FSM variables and their states}
    $G \gets$ parse\_verilog ($V$)\;
    $Regs \gets$ identify\_registers (G)\;
    $result \gets$ $\{\}$\;
    \For{$r \gets$ Regs}{
        $\text{const\_sources} \gets$ get\_const\_sources ($r$)\;
        \If{$|\text{const\_sources}| > 0$ and $\text{has\_control\_loop} (r)$}{
            $result \gets result \cup \{(r, \text{const\_sources}\}$\;
        }
    }
    \Return{result}\;
 \caption{Overall algorithm to extract FSMs}
\end{algorithm}

In \textit{parse\_verilog}, to report all the hierarchical name (handle name) for the FSM variable,
we add a new type of vertex to represent Verilog module and in each vertex we store a
\texttt{parent} to point back to its hierarchical parent.

The algorithm for \texttt{identify\_registers} is shown in Algorithm~\ref{alg:2}, which examines
the incoming edges for each named vertex in the graph $G$. We only care about named vertex because
it is impossible to create a registers without an explicit name in Verilog. This will speed up
the search significantly.
\begin{algorithm}[!tbh]
    \label{alg:2}
    \KwIn{Graph $G$}
    \KwResult{Set of register vertices}
    $Regs \gets$ \{\}\;
    \For{$v \gets$ VertexNamed(G)} {
        $is\_reg \gets true$\;
        \For{$e \gets InComingEdges(v)$} {
            \If{$e.type \neq BlockingAssignment$} {
                $is\_reg \gets false$\;
                \Break\;
            }
        }
        \If{$is\_reg$} {
            $Regs \gets Regs \cup \{v\}$\;
        }
    }
    \Return{Regs}\;
 \caption{Algorithm for \texttt{identify\_registers}}
\end{algorithm}

The algorithm for \texttt{get\_const\_sources} is shown in Algorithm~\ref{alg:3}. We check the
connected vertices and if it is an external port, i.e. IO, we early out the recursion. Notice
that to avoid infinite loop and efficient search, in the actual implementation we need to
combine the algorithm with standard BFS with vertex coloring for visited/unvisited.
\begin{algorithm}[!tbh]
    \label{alg:3}
    \KwIn{Vertex $w$}
    \KwResult{A set of constant sources}
    $result \gets \{\}$\;
    \For{$v \gets ConnectedFrom(w)$} {
        \eIf{$v.type == Constant$} {
            $result \gets result \cup \{v\}$\;
        } {
            \If{$v.type == ExternalPort$} {
                \Return $\{\}$\;
            }
            $r \gets \text{get\_const\_sources}(v)$\;
            \If{$|r| == 0$} {
                \Return $\{\}$\;
            }
            $result \gets result \cup r$;
        }
    }
    \Return{result}\;
 \caption{Algorithm for \texttt{get\_const\_sources}}
\end{algorithm}

The algorithm for \texttt{has\_control\_loop()} is shown in Algorithm~\ref{alg:4}. Notice that
finding all the paths between two arbitrary vertices is hard and memory inefficient. Since the
goal is just to find one loop that contains a control vertex, we can significantly speed up
the search process by borrowing the idea from union-find, as shown between Line~\ref{line:1} and
Line~\ref{line:2}. We maintain a set that contains all the vertex connected by or \textit{indirectly}
connected by a control edge. This is done through two passes: the first pass collect all the
vertices directly connected by a control edge, and the second pass colors any vertex connected by
vertices in the control set. During the second pass, if a new vertex is colored and it has been
visited, we remove it from the visited set so that we can recompute any path from that vertex.
The coloring process ended when there is no vertex let to color or the target vertex has been
colored and we have reached a loop. The correctness of the algorithm can be proved by induction
on the graph, where the base case is the smallest loop with control loop and we add a vertex to
the loop during induction step. For completeness we show the BFS version of the algorithm.
\begin{algorithm}
    \label{alg:4}
    \KwIn{Vertex $r$}
    \KwResult{$true$ if it contains a control vertex connected to $r$}
    $control\_set \gets \{\}$\;
    $visited \gets \{\}$\;
    $working\_set \gets queue\{r\}$\;
    // first pass to get the control set\;
    \While{$|working\_set| > 0$} {
        $v \gets working\_set.pop()$\;
        \eIf{$v \in visited$} {
            \Continue\;
        } {
            $visited \gets visited \cup \{v\}$\;
        }
        \For{$e \in Edges(v)$} {
            $w \gets e.end()$\;
            \If{$e.type == Control$} {
                $control\_set \gets control\_set \cup \{w\}$\;
            }
            $working\_set \gets working\_set \cup \{w\}$\;
        }
    }

    // second pass compute the result\;
    $working\_set \gets = queue{r}$\;
    $visited \gets \{\}$\;
    \While{$|working\_set| > 0$} {
        $v \gets working\_set.pop()$\;
        \If{$v == r$ and $v \in control\_set$} {
            \Return $true$;
        }
        \eIf{$v \in visited$} {
            \Continue\;
        } {
            $visited \gets visited \cup \{v\}$\;
        }
        \For{$w \in EdgesVertex(v)$} {
            \label{line:1}\If{$v \in control\_set$ and $w \in control\_set$} {
                $control\_set \gets control\_set \cup \{w\}$\;
                \If{$w \in visited$} {
                    $visited \gets visited \setminus \{w\}$\;
                }
            \label{line:2}}
            $working\_set.add(w)$\;
        }
    }
    \Return{false}\;
 \caption{Algorithm for \texttt{has\_control\_loop()}}
\end{algorithm}

\subsection{Counter Identification}
The algorithm to detect counter is shown in Algorithm~\ref{alg:5}. We are looking for a pattern of
$v \gets v + constant$ for a given $v$. However, since values may get passed through multiple
vertices, we need to recursively check all the assignment patterns. Notice that for performance,
we can rewrite the algorithm using a while loop instead of a recursive call.

\begin{algorithm}[htb]
    \label{alg:5}
    \KwIn{Vertex $w$}
    \KwResult{$true$ if vertex is a counter}
    \For{$e \gets EdgesTo(w)$} {
        $v \gets e.start()$\;
        \If{$v.type == Assignment$ and $(x, w) \in E(G)$ and $x.type == Addition$} {
            \If{both a constant and $w$ connected to $x$} {
                \Return{true}\;
            }
        }
        $r \gets \text{detect\_counter(v)}$\;
        \If{$r$} {
            \Return{true};
        }
    }
    \Return{false}\;
 \caption{Algorithm for \texttt{is\_counter}}
\end{algorithm}

\subsection{Coupled FSM}
For given two FSM vertices $v_1$ and $v_2$, We can modify Algorithm~\ref{alg:4} where the
terminal condition is $v == v_2$ instead of $v == r$ and the algorithm will work out just
fine. Conceptually, if $v_1$ and $v_2$ are the same we will get a loop, which is what the
algorithm is designed for.

\subsection{FSM Transition Arc Extraction}
Because the comparison nodes connect the inner logic with control edges, we have traverse
the control edges to figure out the state transition logic. Because both explicit FSM and
implicit counters use explicit values to control the logic, such as \texttt{==} or \texttt{<},
we can use the comparison node to compute the state transition as indicated in the dataflow.
As shown in Algorithm~\ref{alg:6}, the main idea is to start from a comparator node where the
FSM register node is used to compare a constant state value, denoted as \texttt{state\_0},
and then traverse to an assignment where the state node is assigned to another constant state
value, denoted as \texttt{state\_1}. Since it is control logic, all the path has to be control
path and taken on the \texttt{true} path.

\begin{algorithm}[htb]
    \label{alg:6}
    \KwIn{State comparator node $v$}
    \KwIn{State assign node $w$}
    \KwResult{$true$ if under condition $v$, $w$ happens}
    \While{$e \gets Traverse(w)$} {
        \If{$e.type() != \texttt{control}$ \text{ or } $e.type() == \texttt{false}$} {
            \text{stop traverse from $e$}\;
        }

        $n \gets e.end()$\;
        \If{$n == w$} {
            \Return{true};
        }
    }
    \Return{false}\;
 \caption{Algorithm for detecting FSM transition arc}
\end{algorithm}

\subsection{Performance Optimization with Parallelism}
Notice that all the algorithms used in Algorithm~\ref{alg:1} are intentionally designed to not
share data during the iteration, nor do they modify the graph structure. As a result, we have
a list of embarrassingly parallel tasks each can be computed independently on each CPU core.
Given the nature of graph size as the design complexity grows, we can implement a thread-pool
and dispatch the tasks to available CPU cores.

\subsection{Parsing Verilog/SystemVerilog Files and Producing Graph}
There is an open-source SystemVerilog parser project called
slang~\footnote{\url{https://github.com/MikePopoloski/slang}}, and we have been contributing
to it by adding serialization ability on the symbol level (one compilation stage after syntax
tree parsing). The serialization dumps a JSON file and we can parse that efficiently using
simdjson~\footnote{\url{https://github.com/simdjson/simdjson}}.

When generating the graph, since we are not interested in the exact logic other than label
the vertex as binary operation, unary operation, etc, we lose lots of logical information
that could be useful for other tasks. However, at this stage the extra information are
unnecessary and waste of memory and thus shall not be kept inside the data structure.


\section{Interaction with Formal Verification Tools via SVA and its Applications}
Once we have obtained FSM information from the dataflow graph, we can generate FSM specification
based on the FSM states, types, interaction, and transition. Although there are many different
languages for specifying formal temporal logic, we choose SystemVerilog Assertions (SVA) for
several reasons:
\begin{itemize}
    \item SVA are widely supported in hardware formal verification tool chains.
    \item Designers can reuse existing input constraints when verifying FSM properties in design.
    \item Can be used for traditional constrained random coverage metric.
\end{itemize}

\subsection{SVA Code Generation Example}
The code generation for SVA from FSM properties are very mechanical and will not be the main
focus of this paper. Listing~\ref{lst:5} shows an example of checking whether a FSM \texttt{state}
is possible to transition from \texttt{STATE\_0} to \texttt{STATE\_1}. Notice that we use
\texttt{cover} so that formal tools can solve for possible transition given constraints.
The SVA files can either be a bind file, or inside top-level test-bench file where the
design under test is instantiated.

\begin{lstlisting}[style={verilog-style}, caption={SVA code example for state transition
    coverage},
    label={lst:5}]
    property fsm_state_0;
      @(posedge clk) state == STATE_0 |=> state == STATE_1;
    endproperty
    FSM_STATE_0: cover property (fsm_state_0);
\end{lstlisting}

After we collect results from formal tools, we can provide an API so that verification
engineers can check against their specification programmatically, for instances, number
of states, possible state transition and so on. Currently we support Cadence JapserGold
and we plan to support CosA2 via yosys.

\subsection{Dead code Detection}
One application for automatic state transition and reachability calculation with the help
of formal tools is dead code detection. If a state or a state transition is proven unreachable,
it usually implicates either dead code or bug in the FSM design. We will continue this section
with a case study on dead code since it is very common in generated code where the generator
parameters are used to produce the states and output logic associated with the state. We will
use RocketChip's DCache as an example here.

The main FSM in DCache is \texttt{release\_state}, where there are 8 pre-defined states. In
generated Verilog, these state values are represented as constants, $0, 1, \dots, 7$. If we
use the default configuration on RocketChip, Chisel will generate an FSM with all 8 states,
nested in a single \texttt{always} block.

Without much discerning, the FSM logic in generated Verilog looks like any other generated
code. However, once we use our FSM extraction algorithm and use JapserGold to verify the
state transition based on the graph analysis, we have discovered a unreachable state,
$4$. This is because given the default configuration, state $4$ is never going to be
assigned to \texttt{release\_state}. However, there are still some downstream logic that
expect \texttt{release\_state} to be $4$ and behave accordingly.

Does this dead state matter in the synthesis tools? Contrary to common belief, the answer
is yes. We compare two version of DCache: DCache-Original and DCache-Improved, where we
manually remove the dead state and its corresponding logic from the latter version. The
edits are formally verified to be functionally equivalent by JasperGold. We
use FreePDK-45 and Design Compiler for synthesis. To ensure a fair comparison, we target
both design at 100MHz clock speed, ultra-high effort, and flatten everything. We also
removed SRAM component to ensure that the comparison only focus the control logic.
Table~\ref{table:1}
shows the final synthesis result.

\begin{table}[!htb]\label{table:1}
    \centering
    \caption{Synthesis result comparison on RocketChip's DCache with and without dead code on
    FreePDK-45 with Design Compiler.}
    \begin{tabular}{|l|l|l|l|l|}
        \hline
        Design Name     & \# of Ports & \# of Nets & \# of Cells & \# of Combinational Cells \\ \hline
        DCache          & 2517        & 10476      & 8469        & 6213                      \\ \hline
        DCache-Improved & 2517        & 10435      & 8367        & 6112                      \\ \hline
    \end{tabular}
\end{table}

Although the difference is minimal, it shows that by properly incorporating dead logic
detection with our FSM extraction algorithm, we can reduce the size of synthesized netlist
without changing the circuit logic.

\end{document}
