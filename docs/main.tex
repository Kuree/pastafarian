\documentclass{article}
\usepackage[utf8]{inputenc}
\usepackage{tikz}
\usepackage{amsmath, amssymb}
\usepackage{xcolor}
\usepackage{listings}
\usepackage[autosize]{dot2texi}
\usepackage[margin=1in]{geometry}

\definecolor{vgreen}{RGB}{104,180,104}
\definecolor{vblue}{RGB}{49,49,255}
\definecolor{vorange}{RGB}{255,143,102}


\lstdefinestyle{verilog-style}
{
    language=Verilog,
    basicstyle=\small\ttfamily,
    keywordstyle=\color{vblue},
    identifierstyle=\color{black},
    commentstyle=\color{vgreen},
    numbers=left,
    numberstyle=\tiny\color{black},
    numbersep=10pt,
    tabsize=8,
    morekeywords={unique, logic, always_ff, typedef, enum}
    literate=*{:}{:}1
}

\title{FSM Extraction Algorithm}
\author{Keyi Zhang}
\begin{document}

\maketitle
\section{Graph Definition for the Circuit}\label{sec:1}
Let the entire circuit design to be represented as a graph, $G(V, E)$. We define $V$
as terminals of a net, and $E$ as relations between two nets terminals.
For instance, in Listing~\ref{lst:1}, we have three variables, \texttt{a}, \texttt{b},
and \texttt{c}, hence we have three \textit{named} vertices: they are named because
they have direct correspondence to Verilog variables. Similar to synthesis, we create
a new vertex for every u-/bi-nary operators. These new vertices are not named
since they are created through expressions. For instance, \texttt{b + c} is represented
as a new vertex connected to both \texttt{b} and \texttt{c}.
For assignment, we introduce additional vertices as
``pass-through'' nodes, i.e. additional node that connects two vertices. For instance,
in Listing~\ref{lst:1}, the assignment node connects \texttt{a} and the new vertex
created by \texttt{b + c}, as showing in Figure~\ref{fig:1}.
\begin{lstlisting}[style={verilog-style}, caption={Graph example from Verilog},
    label={lst:1}]
logic [15:0] a;
logic [15:0] b;
logic [15:0] c;

assign a = b + c;
\end{lstlisting}

\begin{figure}
    \centering
    \begin{tikzpicture}
    \input{figure1.tex}
    \end{tikzpicture}
    \caption{Graph representation for Listing~\ref{lst:1}.}
    \label{fig:1}
\end{figure}

To describe the behavior of conditional logic, e.g. \texttt{case} and \texttt{if}, and
clocking block, such as \texttt{always\_ff}. We introduce a new type of vertex, i.e.,
the \textit{control} vertex, and new type of edge for non-blocking assignment, as
shown in Listing~\ref{lst:2} and Figure~\ref{fig:2}. Notice that by introducing
control vertex and edge, we allow ``influence'' from one vertex to another vertex
even though there is no direct assignment. Mathematically speaking, that means
there is a path from the vertex used for conditional logic to the assigned vertex; in
this case, a path from \texttt{c} to \texttt{a}.


We claim without proof that the graph
constructed by this method is a directed simple graph, which implies that any loop
in the graph has to go through at least two vertices.

We define the \texttt{source} of vertex is the function $source(v)$:
\[
    source(v) = \{x \mid x \in \mathcal{V} \cup {v}\quad\text{for all}\quad x \in \mathcal{E}(v) \} \cup 
                \{source(x) \mid x \not \in \mathcal{V} \cup {v}\quad\text{for all}\quad x \in \mathcal{E}(v)\}
\]
where $\mathcal{V}$ denotes the set of vertices that does not have any incoming edges,
i.e., constants and top-level IO ports, and $\mathcal{E}(v)$ is defined complete
set of $(x, v)$ such that $(x, v) \in E$ and $(x, v)$ is not a control edge.
Notice that this is a recursive definition to search the ``ultimate'' source of the
vertex, including itself.

\begin{lstlisting}[style={verilog-style}, caption={Control vertex and non-blocking
    assignment Verilog code},
    label={lst:2}]
logic clk;
logic [15:0] a;
logic [15:0] b;
logic [15:0] c;

always_ff @(posedge clk) begin
    if (b == c) begin
      a <= c;
    end
end
\end{lstlisting}

\begin{figure}
    \centering
    \begin{tikzpicture}
    \input{figure2.tex}
    \end{tikzpicture}
    \caption{Graph representation for Listing~\ref{lst:2}. Dotted arrow represents edge with control
    type, and dashed arrow represents non-blocking assignment.}
    \label{fig:2}
\end{figure}

\section{Finite State Machine (FSM) Properties and Its Graph Equivalence}
Based on how FSM are used and programmed in Verilog, there are three properties that
defines an FSM:
\begin{enumerate}
    \item The FSM state variable has to be a register.
    \item All the sources to that FSM variable have to be either constant, ot itself,
    or combination of both with binary operators such as addition.
    \item The current value of the FSM state should be used to determined the next
    state.
\end{enumerate}

One caveat of last property is that it excludes the cases where a state machine
advances its state at every cycle regardless of inputs and state. This only happens
when the FSM is used as a counter and the maximum value of the counter is the maximum
value specified by its bit width. This, however, is trivial to detect, as shown
later.

Given the graph construction described in Section~\ref{sec:1}, we can show that the
equivalent following graph properties:
\begin{enumerate}
    \item Given vertex $v$, for all its connected vertices $w \in V$ such that
    $(w, v) \in E$, $E(w, v)$ is non-blocking and $V(w)$ is assignment.
    \item Given vertex $v$, each element in $source(v)$ has be to either a constant,
    or itself ($v$).
    \item Given vertex $v$, for all simple path $P$ from $v$ to $v$,
    $P = (v_0, v_1, \dots, v_i)$ such that $v_0 = v_i = v$ and $(v_{j-1}, v_{j}) \in E$,
    there exist an edge $e = (v_{j-1}, v_j)$ such that e is a control edge.
\end{enumerate}

It is trivial to show the first properties are equivalent since we create an assignment
vertex for each blocking and non-blocking assignment, then connect that assignment vertex
to the sink vertex.

To show that the second properties are equivalent, notice that the $source$ function
is recursive and the base case is the set $\mathcal{V} \cup v$ for a given vertex
$v$. Furthermore, Since $V$ is the set of all constants and IO ports, if $sources(v)$ are itself
or constant, it means that $v$ is not a data-path register.

To show the equivalence of the last properties, notice the introduction of \textit{control}
edge, which indicates the presence of a conditional logic. If a variable's current value
is used to determined the next value, there must be some conditional logic to use the
current value to mux out the next value, hence creating a control edge from that vertex.

\end{document}